\chapter{Introduction and Motivation}
\label{thesis:introduction}
Scalability and improvement of Internet of Things (IoT) devices and protocols
are important research questions.
Low Power Wide Area Networks (LPWANs) technology offers long-range communication
with low poser requirements. Battery powered LPWAN devices can run for years.
For instance, a node sending 100B once a day lasts for 17 years \cite{morin}.
LoRa (short for Long Range) is a spread spectrum modulation
technique, a wireless radio frequency technology for long range and low power platforms
and has become the de facto technology for IoT networks worldwide \cite{what_is_lora}.
LoRaWAN is the open standard backed by the LoRa Alliance. It is a communication protocol and
Medium Access Control (MAC) protocol built on the physical LoRa layer.
LoRaWAN is designed from the bottom up to optimize LPWANs
for battery lifetime, capacity, range, and cost \cite{what_is_lora_wan}.
There are 142 countries with LoRaWAN deployments, 121 network operators,
and 76 LoRa Alliance member operators. Swisscom, Amazon, IBM, CISCO are merely a few of the 
notables LoRa Alliance members \cite{lora_alliance}.
TTN (The Things Network), also a LoRa Alliance member, provides a worldwide LoRaWAN network 
for and from the community. Anyone with a LoRa gateway can register their gateway on TTN, thereby
extending the networks reach. At the time of writing, TTN has 95'208 members, 9'786 gateways, and is
present in 147 countries \cite{ttn}. As LoRaWAN operates in the unlicensed ISM ( Industrial, Scientific and Medical)
radio bands. Therefore no government license is required to operate LoRa devices and gateways. 
This allows hobbyist, enthusiasts, and developers to quickly get started and open networks such as TTN
to grow rapidly.
\\
In a typical LoRaWAN use case, an IoT device such as a sensor sends data out over the air. Then a LoRa gateway picks
the signal up, decodes it, and forwards it over the Internet to the network server which then can send the packet to 
the application server. If needed, a response message can scheduled on the network server who then chooses the best gateway
to send the response back to the IoT device.
LoRa gateways carry the full implementation of the LoRa PHY (the physical layer), the LoRaWAN protocol, as well as 
the packet forwarder. This architecture of the LoRa gateway can be separated and technological stack on the gateway can 
be reduced by running the signal processing functions not on the gateway itself but in a cloud environment. Such a Cloud Radio
Access Network (CRAN) has been previously shown to be beneficial in the 3rd Generation Partnership
Project (3GPP) Long Term Evolution (LTE) \cite{Sousa2016}. The gateway then is left with only minimal functionality it has to support.
As the decoding does not take place on the gateway itself, it does not need do have any LoRa specific hardware e.g the SX1276 transceiver chip
found on LoRa devices and gateways. Rather, the gateway is equipped with an antenna, an amplifier as well as digital to analog (DAC) and 
analog to digital (ADC) converters. On the upstream, the gateway receives LoRa radio signals which it converts into Inphase and Quadrature 
(I/Q) sample stream with the ADC and simply forwards them to the cloud signal processing unit via the internet. 
On the downstream the cloud unit streams a LoRa signal as I/Q samples to the gateway which converts it with the DAC to an analog signal and 
propagates it out over the air. Signals are encoded and decoded on the cloud unit, the Radio Cloud Center (RCC). There are many advantages in such 
a setup but they come at a cost. First advantage is that the gateway can be kept at a much simpler design resulting in significant manufacturing
cost reduction. Also, modifications to the LoRa PHY or LoRaWAN are easier to introduce as the physical layer is implemented in software. 
Gateways that are once deployed do not need to be physically replaced in case of an upgrade as they are agnostic to the underlying protocol and
just convert and transceive (transmit and receive) I/Q samples. Updates to the protocol can be realized with just updating the software implementation. 
A Low Power Network (LPN) provider saves cost by not having to drive out to the deployed gateways throughout the country to upgrade their versions.
The disadvantage is the high throughput of the I/Q samples stream between the gateway and the RCC. Streaming the I/Q samples between gateway and RCC 
has significantly higher bandwidth requirements than just demodulating the signal on the gateway and forwarding the decoded LoRa packet as it is done 
in the non cloudified setup. Cloudifying the LoRa gateways also brings the advantages of setting the base for Software Defined Networking (SDN) and 
Network Function Virutalization (NFV) by centralizing the resources in the RCC that were before distributed on the individual gateways.
Goal of this work is setting up a CRAN architecture for LoRa by simplifying the gateways as described above and moving the signal processing out of the 
gateway into a cloud ready environment i.e., Docker.





\section{Description of Work}
This work first gives a general introduction to LoRa, LoRaWAN and its applications, then dives 
into more details regarding the LoRa physical layer. Then it gives an overview over existing 
software implementations of the LoRa PHY. There are two main contributions. First, this work 
implements a CRAN for LoRa, gives an architectural overview as well as the implementation details.
It evaluates the architectural and network related requirements. We developed a simple protocol in raw LoRa,
meaning not compliant with the LoRaWAN standard, where a hardware IoT device has a queue of packets to transmit then, depending on 
wether it requires an acknowledgment, waits for a few seconds for a response or just transmit the next packet in the queue in an interval.
If the packet required to be acknowledged but no acknowledgment is received, the same packet will put as first item in the queue.
We use this protocol to analyze our CRAN for LoRa architecture.
\\
Second, as the LoRa PHY is closed source, there is no official documentation on how the LoRa PHY is implemented.
The existing implementations are all reverse engineering attempts with various degree of success. 
They all focused first on decoding LoRa signals transmitted by a real LoRa hardware. For the CRAN to work, not only is it 
necessary to decode signals but also the encoding of downstream LoRa gateway signals is required. To achieve this we developed a tool 
that allows the generation of downstream signals in software.


\section{Thesis Outline}
The rest of the thesis is structured the following way.
In chapter~\ref{chap:lora_and_lorawan} an introduction to LoRa and 
LoRaWAN is given and where it fits compared to other wireless technology. 
Then more in depth explanation is given for LoRa signals such as the modulation scheme and key 
factors such as spreading factor coding rate and packet structure.
\\
\\
In the following chapter and overview of current software defined radio implementations for LoRa is given,
how they are implemented as well as their level of sophistication. In this context the GNU Radio 
framework is introduced as well.
\\
\\
In chapter~\ref{chap:cran_in_cellular} the C-RAN architecture for cellular networks 
is introduced to show what steps are needed to move from a traditional architecture to a C-RAN architecture
and what benefits it can bring. 
\\
\\
The next chapter continues with the main part of the thesis, namely the C-RAN for LoRa. 
The architectural overview is given as well as the implementation details for all 
involved components. Then the C-RAN experiment is described where 
network utilization, network delay and processing delay are investigated.
At the end of the chapter the results are presented.
\\
\\
Chapter~\ref{chap:lora_tools} presents some tools that were developed 
during this thesis for encoding and signal visualization that may be helpful
for further developing LoRa processing in software.
\\
\\
This ties directly into the the second to last chapter where the future work 
for a C-RAN for LoRa may lead to. It also list some limitations of this current 
C-RAN architecture that can be improved and further developed.
\\
\\
Finally, the last chapter summarizes the key aspects and concludes this work.


