\chapter{LoRa and LoRaWAN}
LoRa is a modulation technique derived from chirp
spread spectrum technology\cite{what_is_lora}. 
Originally developed by Cycleo, a french company, LoRa has been acquired by Semtech~\cite{limits_lora}.
LoRa signals spread over multiple frequencies using the whole available bandwidth.
This makes the signal more resilient against noise on a disrupting frequency. As LoRa
signal are sent over the unlicensed ISM bands, this resilience is an important factor.
While LoRa is the modulation technique on the physical layer, LoRaWAN on the other hand 
is an open communication protocol backed by the Lora Alliance. LoRaWAN specifies packet format,
duty cycles, key exchanges and many more things needed for an efficient and cooperative LoRa network.
A LoRa network is and LPWAN where battery powered devices can stay operating up to 17 years, making LoRa
a popular choice for IoT devices as shown in the example given in the introduction in chapter~\ref{thesis:introduction}. 
The TTN network for example is used for cattle tracking, smart irrigation as well as smart parking applications~\cite{ttn}.

\begin{figure}[h]
    \centering
    \includegraphics[width=1\textwidth]{figures/LoRa_context.png}
    \caption{LoRa vs other wireless technology\cite{lora_context}}
    \label{fig:lora_context}
\end{figure}

Figure~\ref{fig:lora_context} shows LoRa compared two other wireless technologies, Wi-Fi and cellular. Both Wi-Fi and cellular
are high in bandwidth with cellular having a longer range than Wi-Fi. They both have a much higher power consumption compared to LoRa.
LoRa has lower bandwidth but a high range. In a experiment during a TTN conference LoRa signals from a low orbit satellite were received~\cite{loa_satellite}.
On the other hand, as LoRa is designed for long range and low power, only few bytes are transmitted per day while Wi-Fi and cellular are capable of video streaming.
In urban areas LoRa has a range of 2-5 km and 15 km in suburban areas~\cite{limits_lora}.\\
LoRaWAN is not the same all around the world. There are regional parameters that come into play, one is for example the frequency band.
In Europe LoRaWAN operates on the in the 863-870MHz and 433MHz ISM band and in North America the 902-928MHz ISM band. Also channel bandwidth and maximum transmission 
settings are regulated by the government and thus are not the same for all regions~\cite{lora_wan_regional}.


\section{LoRaWAN architecture}
A LoRaWAN network architecture is a star-of-stars topology. The gateways relay the messages between the end-devices and a central network server.
Gateways are connected to the network server via IP connections, converting the RF packets to IP packets and vice versa~\cite{about_lora_wan}.
Network nodes are not associated with a specific gateway, rather messages sent by a node can be received by multiple gateways. Each gateway will then 
forward the the message to the network server who does the complex things such as filtering redundant packages, security checks, forwarding the messages
to the right application server etc.~\cite{what_is_lora_wan}.
As network communication is bidirectional, the network server is also responsible for scheduling responses to the end-nodes. There are different classes of 
end-nodes which will be described in the next section.

\begin{figure}[h]
    \centering
    \includegraphics[width=1\textwidth]{figures/lorawan_network.png}
    \caption{LoRaWAN network architecture~\cite{what_is_lora_wan}}
    \label{fig:lorawan_network}
\end{figure}

As depicted in Figure~\ref{fig:lorawan_network}, the packets sent by end-devices (on the far left) such as alarms, tracking devices and monitoring devices,
can be received by multiple gateways. As the end-nodes are not linked to a particular gateway, the can be moved freely which is an important requirement for 
assets tracking.\\
The Figure also shows how security is built into LoRaWAN. The payload is end-to-end encrypted from the end-nodes to the applications server.
A unique 128-bit network session key is shared between  the end-device and the network server and 
another 128-bit application session key is shared end-to-end at the application level~\cite{about_lora_wan}.
With those measures LoRaWAN prevents eavesdropping. Spoofing is prevented by a MIC (Message Integrity Code)
in the MAC payload, and replay attacks are prevent by utilizing frame counters~\cite{lora_security}.


\section{End-node Classes}
There are three classes of end-devices. The following description is adapted from the LoRa Alliance guide~\cite{about_lora_wan,what_is_lora}:
\begin{itemize}
    \item Class A, Lowest power, bi-directional end-devices:\\
    \\
    This is the default class, supported by all LoRaWAN devices.
    It is always the end-node that initiates the communication. After an uplink
    two downlink windows open for the end-device to receive a response, enabling bi-directional communication.
    Either the first is used, or the second, but not both receive windows.
    The end-device can rest in low-power sleep mode, wake up when it needs to send a packet, receive a response
    in the downlink window, then go back to seep. This is an ALOHA-type of protocol. Class A devices have the lowest 
    power consumption. Downlinks from the server have to wait for an uplink from end-device and cannot be initiated directly.
    \item Class B, Bi-directional end-devices with deterministic downlink latency:\\
    \\ 
    Additionally to Class A receive windows, a Class B device opens extra receive windows at scheduled times.
    This is achieved by time-synchronized beacons from the gateway to the end-device to notify the end-device
    to open a receive window.

    \item Class C, Lowest latency, bi-directional end-devices:\\
    \\
    Devices of this class have always open receive windows, except for when they are themselves transmitting.
    A downlink transmission can be initiated by the network server at any time (assuming the device is not currently transmitting)
    resulting in no latency. Class C devices however use the most energy. They are more suitable for plugged in devices rather than
    battery powered devices.

\end{itemize}

\newpage


\section{LoRa signal (uplink)}
\subsection{Chirps}
A LoRa signal is a series of so called chirps as LoRa is derived from the Chirp Spread Spectrum modulation (CSS) technique. 
There are up-chirps and down-chirps. In CSS chirps are deliberately spread across the available bandwidth. Up-chirps go from low frequency 
to high frequency and down-chirps go from high frequency to low frequency. In Europe the LoRaWAN bandwidth for is 125 kHz. Assuming a center
frequency of 868.5 MHz, which is in the european ISM band, a full up-chirp, so called sweep, would go from 868.4375 MHz to 868.5625 MHz.

\begin{figure}[h]
    \centering
    \includegraphics[width=0.6\textwidth]{figures/chirp_mobilefish.png}
    \caption{Up- and down chirps~\cite{lora_chirp_mobilefish}}
    \label{fig:chirp_mobilefish}
\end{figure}

Figure \ref{fig:chirp_mobilefish} shows the linear frequency increase resp. decrease over time over the full bandwidth
for up-chirps and down-chirps. Data is encoded by frequency jumps in the chirps. 

\begin{figure}[h]
    \centering
    \includegraphics[width=1\textwidth]{figures/signal_Goodbye!_SF9_CR4_5.png}
    \caption{Own recording of uplink transmission by arduino equipped with a LoRa shield}
    \label{fig:Goodbye}
\end{figure}

The LoRa signal shown in~\ref{fig:Goodbye} carries the message  "Goodbye !". This message was sent 
with a spreading factor (SF) of 9 and coding rate of 4/5. The terms spreading factor and coding rate 
will be discussed later on.\\
As one can see, a typical LoRa signal start with a so called preamble, which are the 10 up-chirps at 
the beginning. Those are followed by two down-chirps, which signify the end of the preamble and the start 
of the actual payload. In this payload is a header, the actual encoded message followed by a Cyclic Redundancy Check (CRC).
The CRC is used for error correction.
\subsection{Symbol and Spreading Factor}
A LoRa signal holds various symbols. A symbol encodes one or more bits of data.
The spreading factor determines the number of encoded bits in a symbol.
In the shown recording one symbol holds 9 bits of data as the spreading factor of that signal was set to 9.
It follows that a symbol has $2^{SF}$ values. Those values range from 0 to 511 in case of SF 9. A sweep signal of SF 9
thus has 512 chips (no to be confused with chirps)~\cite{lora_symbol_mobilefish}.
The chips go linearly from low to high and then wrap around once the maximum frequency is reached.
\\
In Figure~\ref{fig:fict_symbols} a fictional symbol with SF 7 is shown. This particular 
arrangement of chips highlighted in orange would denote the symbol "1011111". Those 7 bits correspond 
to the decimal value 95.
\begin{figure}[h]
    \centering
    \includegraphics[width=1\textwidth]{figures/chips_and_symbols.png}
    \caption{Chips and symbols value~\cite{lora_symbol_mobilefish}}
    \label{fig:fict_symbols}
\end{figure}

In Figure \ref{fig:Goodbye_decoded}, a real world example is shown. The same LoRa signal as in Figure~\ref{fig:Goodbye} with SF 9 with the message "Goodbye !"
run through modified version of the LoRa decoder by Robyns et al.~\cite{robyns} and then through a python script where we match the samples to the symbols and their 
values. The last symbols encodes the hex value 142 which corresponds to these 9 bits "101000010". In a SF 9 signal each symbol encodes 9 bits.
\begin{figure}[h]
    \centering
    \includegraphics[width=1\textwidth]{figures/Goodbye_decoded.png}
    \caption{Running the signal through our toolchain, matching symbols with samples}
    \label{fig:Goodbye_decoded}
\end{figure}

\newpage

\subsection{Coding Rate}
LoRa signals are encoded with a coding rate (CR). The CR denotes the proportion of how many bits carry actual information. The bits that do not carry information are used 
for Forward Error Correction. The formula for coding rate is $CR=4/(4 + CR)$ where CR $\in \{1,2,3,4\}$. A CR of 1 is thus the proportion of 4/5 of actual information over 
bits used for error correction\cite{SX_design_guide,coding_rate_mobilefish}.
With FEC, corrupted bits e.g. due to interference can be corrected. With CR of 4, corresponds to $4/8 = 1/2$, half the transmitted bits carry information, the other half is for FEC.
The higher the CR (from 1-4) the more bits can get corrupted and corrected by FEC.
On the other hand, the higher the CR the more bits need to be transmitted which drains the battery more.


\subsection{Spreading Factor \& Time on Air}
The longer the packet, the longer the transmission time. LoRa packets can be shortened
by omitting the optional CRC field or sending packets with implicit header mode where the
no header is sent and the settings that would have been specified in the header have to be 
predefined manually on the end device.\\
Assuming constant packet size and same bandwidth, varying the spreading factor increases resp. decreases the time on air.
The higher the SF, the longer the time on air. Higher SF means longer range. The spreading factor goes from 7 to 12. SF 7 has 
the shortest range, SF 12 the longest. The spreading factor essentially sets the duration of a chirp, a full sweep~\cite{exploratory_eng}.
\\
The symbol time is defined in the LoRa Design guide by $T_{sym} = \frac{2^{SF}}{BW}$~\cite{SX_design_guide}.
It follows as stated above, that the higher the SF the longer the symbol duration. Also, the higher the bandwidth (BW) the shorter
the symbol duration. In Europe the BW is 125~kHz, while in North America a BW of 500~kHz is allowed.
It also follows that with an increase in SF by 1 the symbol duration is doubled. The bit rate $R_{b}$  is then defined by 
$R_{b} = SF*\frac{[\frac{4}{4 * CR}]}{[\frac{2^{SF}}{BW}]}$ with CR being the coding rate for the error correction scheme~\cite{lora_modulation_basics}.
It follows from the formula that the higher the coding rate the lower the bit rate as with a higher CR more redundancy is added 
for the error correction scheme. Highest data rate for $BW=125~kHz$ and $CR=1$ is achieved 
with SF 7 resulting in a data rate of 5.5 kbits/s and the lowest data rate is achieved with SF 12 resulting in a data rate 0.29 kbits/s.
\\
The spreading factors are orthogonal to each other, meaning signals on different spreading factors do not interfere with each other. This is 
Code Division Multiple Access (CDMA) where a shared medium i.e. the bandwidth is optimized for multiple access.


\chapter{LoRa in SDRs}
josh blum
matt knight
pieter robyns

\section{C-RAN in LTE}
advantages
graphics