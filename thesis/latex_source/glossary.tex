\chapter*{Glossary}
\addcontentsline{toc}{chapter}{Glossary}
\markboth{GLOSSARY}{}


\begin{description}
  \item[C-RAN] Cloud / Centralized Radio Access Networks centralize the BBU components of the radio towers in a BBU hotel.
  \item[LoRa] a spread spectrum modulation technique derived from chirp spread spectrum (CSS) technology.
  \item[LoRaWAN] the open communication protocol based on LoRa backed by the LoRa Alliance 
  \item[LoRa Alliance] A non-profit association committed to the development and promotion of the open LoRaWAN standard.
  \item[PHY] is the physical layer of LoRa which is proprietary and belongs to Semtech. Software implementations of LoRa try to reverse engineer the PHY.
  \item[RF Technology] All technology related to radio frequency.
  \item[TCP / IP] The Transmission Control Protocol and Internet Protocol operate on the Transport layer resp. the Internet layer and is used in local networks as well as on the Internet.
  TCP controls how data is transmitted while IP is responsible for addressing and routing. 
  \item[TX] Concerning the transmission of data.
  \item[SDR] Software Defined Radio allow the signal processing to be done in software on general purpose hardware.
  \item[Socket] are system resource for receiving and sending data in a computer network.
  \item[RX] Concerning the reception of data   
\end{description}
