\section{C-RAN for LoRa}\label{c-ran-for-lora}

An arduino with a LoRa shield sends out packets over the air in an
interval. Some packets require an acknowledgment (ACK). If an ACK is
required, the arduino waits for a certain amount of time for the ACK. If
the ACK arrives in time, the arduino starts transmitting the next
packet. If not, the arduino will resend the packet and again wait for
the ACK.

The RRH (Remote Radio Head) receives radio waves with a LimeSDR. The RRH
streams the IQ samples over the network the the BBU (Base Band Unit).

The BBU decodes the message. If the message says it require and ACK, the
BBU send out IQ samples of the ACK message over the network to the RRH
which transmits them back over the air to the arduino.

\subsection{Run with Docker}\label{run-with-docker}

\begin{enumerate}
\def\labelenumi{\arabic{enumi}.}
\tightlist
\item
  Clone the repo
\item
  Go to the docker directory
\end{enumerate}

\subsubsection{\texorpdfstring{\textbf{\emph{Info}}}{Info}}\label{info}

\begin{itemize}
\tightlist
\item
  The container run in priviledged mode to easily access plugged in USB
  devices
\item
  The container run in network mode host (No NAT or Bridge has to be
  considered). This means the containers have the ip address of the host
  machine. If RRH and BBU run on different machines, find out their
  respective IP with \emph{ifconfig} and pass the address as arguments
  in the docker-compose.yml, see below.
\end{itemize}

\begin{center}\rule{0.5\linewidth}{\linethickness}\end{center}

\subsection{RRH}\label{rrh}

In the RRH directory run:

\begin{verbatim}
docker-compose up 
\end{verbatim}

This starts the Remote Radio Head. The RRH looks for a LimeSDR, it
prints errors if it cannot find one. You can plug one in after the
container has started and it should get detectet. By default it uses the
first LimeSDR it can find.

\subsubsection{Parameters}\label{parameters}

There are various parameters which you can specify in the
\emph{docker-compose.yml} file.

Run this to see what the possible params are:

\begin{verbatim}
./zero_mq_split_a.py -h
\end{verbatim}

Output:

\begin{verbatim}
Usage: zero_mq_split_a.py: [options]

Options:
  -h, --help            show this help message and exit
  --RX-device-serial=RX_DEVICE_SERIAL
                        Set RX_device_serial [default=]
  --TX-device-serial=TX_DEVICE_SERIAL
                        Set TX_device_serial [default=]
  --capture-freq=CAPTURE_FREQ
                        Set capture_freq [default=868.5M]
  --samp-rate=SAMP_RATE
                        Set samp_rate [default=1.0M]
  --zmq-address-iq-in=ZMQ_ADDRESS_IQ_IN
                        Set zmq_address_iq_in [default=tcp://127.0.0.1:5052]
  --zmq-address-iq-out=ZMQ_ADDRESS_IQ_OUT
                        Set zmq_address_iq_out [default=tcp://*:5051]
\end{verbatim}

\begin{longtable}[]{@{}ll@{}}
\toprule
\begin{minipage}[b]{0.18\columnwidth}\raggedright\strut
Param\strut
\end{minipage} & \begin{minipage}[b]{0.18\columnwidth}\raggedright\strut
Explanation\strut
\end{minipage}\tabularnewline
\midrule
\endhead
\begin{minipage}[t]{0.18\columnwidth}\raggedright\strut
RX-device-serial\strut
\end{minipage} & \begin{minipage}[t]{0.18\columnwidth}\raggedright\strut
By default, the program will use the first LimeSDR it can find for
receiving and transmitting signal. If you have two devices you can
specify which should receive by passing the device Serial (See section
\textbf{Help} for more info)\strut
\end{minipage}\tabularnewline
\begin{minipage}[t]{0.18\columnwidth}\raggedright\strut
TX-device-serial\strut
\end{minipage} & \begin{minipage}[t]{0.18\columnwidth}\raggedright\strut
By default, the program will use the first LimeSDR it can find for
receiving and transmitting signal. If you have two devices you can
specify which should transmit by passing the device Serial (See section
\textbf{Help} for more info)\strut
\end{minipage}\tabularnewline
\begin{minipage}[t]{0.18\columnwidth}\raggedright\strut
capture-freq\strut
\end{minipage} & \begin{minipage}[t]{0.18\columnwidth}\raggedright\strut
The frequency in Hz at which the RRH listens for signals. Default value
is 86850000\strut
\end{minipage}\tabularnewline
\begin{minipage}[t]{0.18\columnwidth}\raggedright\strut
samp-rate\strut
\end{minipage} & \begin{minipage}[t]{0.18\columnwidth}\raggedright\strut
How many samples per second. Default value is 1000000. Must be at least
double the bandwidth of the expected signal see \emph{Nyquist-Shannon
principle}\strut
\end{minipage}\tabularnewline
\begin{minipage}[t]{0.18\columnwidth}\raggedright\strut
zmq-address-iq-in\strut
\end{minipage} & \begin{minipage}[t]{0.18\columnwidth}\raggedright\strut
ZMQ address to which the RRH subscribes to receive an IQ samples stream
(from the BBU) to then send out (TX). Default value is
tcp://127.0.0.1:5052 meaning the IQ samples are expected to come from
localhost on port 5052. Normally RRH and BBU are on different devices
but on the same network\strut
\end{minipage}\tabularnewline
\begin{minipage}[t]{0.18\columnwidth}\raggedright\strut
--zmq-address-iq-out\strut
\end{minipage} & \begin{minipage}[t]{0.18\columnwidth}\raggedright\strut
ZMQ address on which the RRH streams out the IQ samples (to the BBU) it
receives (RX). Default is tcp://*:5051 meaning it publishes the stream
on all interface on port 5051\strut
\end{minipage}\tabularnewline
\bottomrule
\end{longtable}

\begin{center}\rule{0.5\linewidth}{\linethickness}\end{center}

To pass the parameters you have to specify them in the
docker-compose.yml

Example:

To pass a capture frequencey of 915M and a sample rate of 250k enter the
params in the following way in the command field:

\emph{docker-compose.yml}

\begin{verbatim}
version: '3'
services:
    rrh:
        build: .
        privileged: true
        network_mode: host
        volumes:
                - /dev/bus/usb:/dev/bus/usb
        command: ["--capture-freq", "915000000", "--samp_rate", "250000"]
\end{verbatim}

\begin{center}\rule{0.5\linewidth}{\linethickness}\end{center}

\subsection{BBU}\label{bbu}

The BBU has two components: * LoRa\_Decoder: receives a stream of IQ
samples from the RRH, decodes the LoRa signal and sends the decoded
message out on a UDP socket * LoRa\_Network\_Server: receives the
messages from that UDP socket and, depending on message content, streams
response IQ samples to the RRH or does not give a response

In the BBU directory run:

\begin{verbatim}
docker-compose up 
\end{verbatim}

This starts both components of the BBU

\subsubsection{Params}\label{params}

The LoRa\_Decoder has the following params:

\begin{verbatim}
Usage: zero_mq_split_b.py: [options]

Options:
  -h, --help            show this help message and exit
  --bandwidth=BANDWIDTH
                        Set bandwidth [default=125000]
  --capture-freq=CAPTURE_FREQ
                        Set capture_freq [default=868.5M]
  --decoded-out-port=DECODED_OUT_PORT
                        Set decoded_out_port [default=40868]
  --samp-rate=SAMP_RATE
                        Set samp_rate [default=1.0M]
  --spreading-factor=SPREADING_FACTOR
                        Set spreading_factor [default=12]
  --zmq-address-iq-in=ZMQ_ADDRESS_IQ_IN
                        Set zmq_address_iq_in [default=tcp://127.0.0.1:5051]
\end{verbatim}

\begin{longtable}[]{@{}ll@{}}
\toprule
\begin{minipage}[b]{0.18\columnwidth}\raggedright\strut
Param\strut
\end{minipage} & \begin{minipage}[b]{0.18\columnwidth}\raggedright\strut
Explanation\strut
\end{minipage}\tabularnewline
\midrule
\endhead
\begin{minipage}[t]{0.18\columnwidth}\raggedright\strut
bandwith\strut
\end{minipage} & \begin{minipage}[t]{0.18\columnwidth}\raggedright\strut
The bandwidth in Hz of the LoRa signal. Default is 125000.\strut
\end{minipage}\tabularnewline
\begin{minipage}[t]{0.18\columnwidth}\raggedright\strut
capture-freq\strut
\end{minipage} & \begin{minipage}[t]{0.18\columnwidth}\raggedright\strut
The frequency in Hz of the LoRa signal. The RRH of course must also
listen on this frequeny. Default is 868500000.\strut
\end{minipage}\tabularnewline
\begin{minipage}[t]{0.18\columnwidth}\raggedright\strut
decoded-out-port\strut
\end{minipage} & \begin{minipage}[t]{0.18\columnwidth}\raggedright\strut
On which port the decoded messages will be sent out. Localhost only. The
LoRa\_Network\_Server needs to be configured to listen on this port.
Default is 40868.\strut
\end{minipage}\tabularnewline
\begin{minipage}[t]{0.18\columnwidth}\raggedright\strut
samp-rate\strut
\end{minipage} & \begin{minipage}[t]{0.18\columnwidth}\raggedright\strut
How many samples per second to expect from the RRH. Default is
1000000\strut
\end{minipage}\tabularnewline
\begin{minipage}[t]{0.18\columnwidth}\raggedright\strut
spreading-factor\strut
\end{minipage} & \begin{minipage}[t]{0.18\columnwidth}\raggedright\strut
The spreading factor of the incoming LoRa signal. From {[}7-12{]}
inclusive. Default is 12\strut
\end{minipage}\tabularnewline
\begin{minipage}[t]{0.18\columnwidth}\raggedright\strut
--zmq-address-iq-in\strut
\end{minipage} & \begin{minipage}[t]{0.18\columnwidth}\raggedright\strut
ZMQ address to which the BBU subscribes to receive an IQ samples stream
(from the RRH) to decode. Default value is tcp://127.0.0.1:5051 meaning
the IQ samples are expected to come from localhost on port 5051.
Normally RRH and BBU are on different devices but on the same
network\strut
\end{minipage}\tabularnewline
\bottomrule
\end{longtable}

\begin{center}\rule{0.5\linewidth}{\linethickness}\end{center}

The LoRa\_Network\_Server has the following params:

\begin{verbatim}
usage: lora_socket_server.py [-h] [-o OUT_PORT] [-i INPUT_PORT]

Connect to udp port for receiving decoded LoRa signals, if an ACK is required
publish ACK iq samples via zmq socket for Remote Radio Head to receive and
send out (TX).

optional arguments:
  -h, --help            show this help message and exit
  -o OUT_PORT, --out-port OUT_PORT
                        zmq port to publish downstream (i.e ACK) iq samples
                        (default: 5052)
  -i INPUT_PORT, --input-port INPUT_PORT
                        UDP port to connect for receiving decoded lora
                        messages (default: 40868)

\end{verbatim}

\begin{longtable}[]{@{}ll@{}}
\toprule
\begin{minipage}[b]{0.18\columnwidth}\raggedright\strut
Param\strut
\end{minipage} & \begin{minipage}[b]{0.18\columnwidth}\raggedright\strut
Explanation\strut
\end{minipage}\tabularnewline
\midrule
\endhead
\begin{minipage}[t]{0.18\columnwidth}\raggedright\strut
out-port\strut
\end{minipage} & \begin{minipage}[t]{0.18\columnwidth}\raggedright\strut
Publish the response IQ samples on all interface on this port. Default
is 5052. (The response is 3 bytes long (``ACK'') and SF 12. This is
hardcoded for now)\strut
\end{minipage}\tabularnewline
\begin{minipage}[t]{0.18\columnwidth}\raggedright\strut
input-port\strut
\end{minipage} & \begin{minipage}[t]{0.18\columnwidth}\raggedright\strut
UDP port to receive the decoded messages sent by the LoRa\_Decoder.
Default is 40868\strut
\end{minipage}\tabularnewline
\bottomrule
\end{longtable}

\begin{center}\rule{0.5\linewidth}{\linethickness}\end{center}

To pass the parameters you have to specify them in the
docker-compose.yml file.

Example:

To have the LoRa\_Decoder send the decoded messages out on port 30300
and the Lora\_Network\_Server to listen on port 30300 accordingly pass
the arguments like below to the respective command field:

\emph{docker-compose.yml}

\begin{verbatim}
version: '3'
services:
        lora_decoder:
                build: ./LoRa_Decoder
                network_mode: host
                tty: true
                command: ["--decoded-out-port", "30300"]
        lora_network_server:
                build: ./LoRa_Network_Server
                network_mode: host
                tty: true
                command: ["--input-port", "30300"] 
\end{verbatim}

\subsection{LimeSDR}\label{limesdr}

\begin{itemize}
\tightlist
\item
  Plug in the antennas on the LimeSDR board on \emph{RX1\_L} and
  \emph{TX1\_1}
\end{itemize}

\subsection{Help}\label{help}

\begin{itemize}
\item
  LimeSDR calibration/gain error:
\item
  \href{https://wiki.myriadrf.org/Lime_Suite}{Download LimeSuite
  Toolkit} to calibrate the LimeSDR
\item
  LimeSDR find device serial:
\item
  With LimeSuite installed run \emph{LimeUtil --find}
\item
  Or run \emph{lsusb -v} and look for the LimeSDR device
\end{itemize}

\begin{center}\rule{0.5\linewidth}{\linethickness}\end{center}

\section{Arduino}\label{arduino}

\textbf{The arduino-lmic library is required
\href{https://github.com/matthijskooijman/arduino-lmic}{Instructions
here}}

\begin{enumerate}
\def\labelenumi{\arabic{enumi}.}
\tightlist
\item
  Go to the arduino directory.
\item
  Compile and upload the code to the arduino
\item
  The arduino runs the protocol in the manner described at the
  beginning.
\item
  It send packets with SF12 and expects the ACK response to be SF12 as
  well.
\item
  After 3 packets the arduino has finished.
\item
  Look at the Serial output for details. Baud rate 9600
\end{enumerate}

\textbf{Info}

PlatformIO was used to compile and upload the image to the arduino.

\begin{center}\rule{0.5\linewidth}{\linethickness}\end{center}

\subsection{Manual installation
Ubuntu}\label{manual-installation-ubuntu}

Visit this guide for
\href{https://wiki.myriadrf.org/Gr-limesdr_Plugin_for_GNURadio}{installing
LimeSDR Plugin for GNU Radio} for more detail. This guide only has the
short version.

Install dependencies for signal processing:

\begin{verbatim}
sudo apt-get update && sudo apt-get install -y gnuradio=3.7.11-10 libboost-all-dev swig git cmake software-properties-common \
libcppunit-1.14-0 libfftw3-bin libvolk1-bin liblog4cpp5v5 python libliquid1d libliquid-dev python-pip \
&& pip install numpy && pip install scipy
\end{verbatim}

Install LimeSuite

\begin{verbatim}
sudo add-apt-repository -y ppa:myriadrf/drivers && sudo apt-get update \
&& sudo apt-get install -y limesuite liblimesuite-dev limesuite-udev limesuite-images \
soapysdr-tools soapysdr-module-lms7
\end{verbatim}

Clone and install LimeSDR Plugin for GNU Radio:

\begin{verbatim}
git clone https://github.com/myriadrf/gr-limesdr && cd gr-limesdr && mkdir build && cd build && cmake .. && make && sudo make install && sudo ldconfig
\end{verbatim}

Clone and install rpp0's LoRa decoder for gnuradio

\begin{verbatim}
git clone https://github.com/rpp0/gr-lora.git && cd gr-lora && git checkout b1d38fab9032a52eaf31bf33a145df45fce7512f\
&& mkdir build && cd build \
&& cmake .. && make && sudo make install \
&& cd .. && rm -rf build \
&& git checkout -b encoder origin/encoder && git checkout 3c9a63f1d148592df2b715496c67ccbc2939ad0d \
&& mkdir build && cd build \
&& cmake .. && make && sudo make install && sudo ldconfig
\end{verbatim}

With pip for python2 install the zmq package:

\begin{verbatim}
pip install pyzmq==18.1.0
\end{verbatim}

Then open the \emph{zero\_mq\_split\_a.grc} and the
\emph{zero\_mq\_split\_b.grc} file in the docker/RRH directory resp. in
the docker/BBU/LoRa\_Decoder directory. Or run the
\emph{zero\_mq\_split\_a.py} resp. the \emph{zero\_mq\_split\_b.py}
script in those directories with your shell. Also run the
\emph{lora\_socket\_server.py} sript inside
docker/BBU/LoRa\_Network\_Server with your shell.

\section{Tools}\label{tools}

In the tools directory in the Encode and Decode directory are multiple
usefuls scripts for encoding and decoding lora without gnuradio

\begin{enumerate}
\def\labelenumi{\arabic{enumi}.}
\item
  First, after you recorded a signal trim the signal with a tool like
  audacity. Else if you want to visualize it with plot\_signal.py the
  signal is shrunk too much to make it fit in the plot.
\item
  After trimming, channelize the signal else the decoder cannot properly
  decode the signal. Run channelizer.py -h to see the options. It takes
  an signal recording via the --input-file option and outputs the
  channelized file as ``channelized.raw''. Don't forget to specify
  bandwidth and sample rate if they differ from the set default values.
\item
  The channelized signal can the be passed to the decoder. The decoder
  prints out the decoded signal and generates a csv file (words.csv)
  containing the words at each sample. Don't forget to specify bandwidth
  and sample rate etc if they differ from the set default values.
\item
  This csv file can be passed to plot\_signal.py which draws the signal
  and the words in the csv file to a pdf (rawframe.pdf). Don't forget to
  specify bandwidth and sample rate if they differ from the set default
  values.
\end{enumerate}

Use the encoder to generate samples for the test\_packet{[}{]} uint8
array in the code. The samples are written to the fiel ``output.bin''

Use the two scripts decoder\_build.sh and encoder\_build.sh to compile
the encode.cc and decode.cc files.

Use VsCode to open the directory ``Encode and Decode'' to have
predefiend debug configurations. The folder `.vscode' has been commited
in this repo.

All recorded uplink signals have been recorded with sample rate 1Million
and transmitted with a bandwidth of 125'000

The decoder only works for signals with an explicit header.
