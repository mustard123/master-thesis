\chapter{Future Work}
The design for a C-RAN for LoRa depends on level of sophistication of
software processing for LoRa. \\
Best case scenario would haven been if software decoding, encoding and the LoRaWan protocol
were already implemented for software defined radios.
\\
As this was not the case, we first had to spend a significant amount of time 
to understand the inner workings of LoRa better. This is made more difficult by the fact 
that the PHY layer of LoRa is proprietary and no official documentation exists.

\section{Limitations}
A first limitation is that the response signal is a recording of a real downlink signal and not 
purely generated in software on demand. Nevertheless we developed the tools to make it possible.
\\
Second, the LoRa protocol implemented on the Arduino is "raw" LoRa and not compliant 
with the LoRaWan standard. This was necessary as decoding a LoRaWan packet is not yet implement in 
the decoder, in terms of destructuring the received packet according to all the LoRaWan fields as shown
in Figure~\ref{fig:lora_wan_struct}.
\\
Third, the experiment, C-RAN architecture and Arduino implementation is based on the assumption 
of a single end-device and gateway in the network. In a LoRaWan, end-devices ignore downlink packets
not destined for them. Our Arduino however happily accepts any "ACK" downlink because of the premise that 
all downlinks are destined for it as it is the only device in the network.\\
Further in a network with multiple gateways, an uplink signal gets picked up by multiple gateways and 
forwarded to the network server. Then the network server discards duplicates and decides over which gateway
the downlink response is sent. In our network however there is only one RRH.

\section{Improvements}
Improvements to the protocol implementation can be made. Though, instead
of improving this protocol, it makes far more sense to adopt the LoRaWan protocol.
The LoRaWan protocol is an open protocol. Implementing it on the decoder requires the 
necessary time and knowledge. The Arduino could already speak the LoRaWan protocol with 
the help of the LMIC library which offers a fairly complete LoRaWan Class A device implementation.
\\
On network server side, an initial improvement would be to dynamically generate the response signal
by incorporating the encoder for downlink signals in the architecture.
\\ \\
Further, there is only one decoder / BBU running because there is only one Arduino sending on one channel.
As the BBU is containerized, the experiment can be run with multiple Arduino devices on different 
channels and spreading factor. Witch Docker compose, multiple BBUs could be started to handle the different 
channels and spreading factor, they would then forward the decoded messages to the network server.
This can be already done with the current implementation. The network server however could be replaced 
with the open implementation of the chirpstack network server, \url{https://www.chirpstack.io/network-server/overview/}. 
It handles among other things the de-duplication of LoRaWan messages and downlink scheduling.
It takes LoRaWan messages from an MQTT broker. MQTT like ZMQ facilitates architectures based on a publish-subscribe strategy. So instead of the decoder sending the decoded messages 
out over an UDP socket and our Python script subscribing, the decoder can publish the decoded message to the 
MQTT broker. For downlink scheduling the chirpstack network server sends the message also over MQTT
back to the gateway for modulation and transmission. For the C-RAN architecture where the downlink is modulated 
in software and not on the gateway anymore, this would need a redesign where the RRH simply receives the I/Q samples
to transmit without having to do any LoRa modulation. Then the chirpstack application server could be 
added for free, giving a nice web-interface similar to the TTN interface, but instead of 
traditional LoRaWan network architecture a C-RAN LoRaWan architecture is running.
\\ \\
For further developing the centralized / cloud aspect of C-RAN, the BBUs instead of running on a single host
controlled by Docker compose could be spread of multiple machines with Docker swarm or Kubernetes. 
Further down the road BBU and network server could be offered as a service via software such as Openstack 
essentially providing C-RAN as a service as demonstrated for LTE by in~\cite{Nikaein2015}.
\\ \\
Finally, it needs to be investigated why the decoder cannot successfully decode I/Q samples of a single signal streamed over and ethernet connection as TCP/IP packets 
if a network delay of more than 500ms is emulated on the connection. Also, there are some open issues on the 
GitHub repo of the decoder and a few items that are not yet implemented such as multichannel decoding and 
CRC checks for payload and header. However, the decoder itself is not in scope of this thesis and is just mentioned here for 
completeness.


