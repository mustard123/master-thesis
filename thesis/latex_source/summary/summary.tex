\chapter{Summary and Conclusions}
Moving to a C-RAN approach for LoRaWAN networks is feasible. For this the
the LoRa gateways functionality must be split up in a into a RRH and BBU component.
It shows how the BBU part can be accomplished in software and the benefits it brings
by containerizing and running it in a virtualized environment. Tradeoff for this 
simplification of the LoRa gateway is the introduction of significant amount of data traffic between
the RRH and the BBU component and a more complex architecture as the one gateway component is now 
split into two components. This offers more flexibility and convenience in the long run but needs to 
be managed and maintained nevertheless. Generally, C-RAN for LoRa brings much of the same advantages as 
a C-RAN for LTE does in terms of CAPEX and OPEX. For at least class A devices, the LoRaWAN protocol allows for more leeway in 
processing as receive windows for downlinks signals are open either 1 or 2 seconds after transmission.
Data traffic between RRH and BBU can be reduced by applying the Nyquist-Shannon sampling theorem. 
By reducing the sampling rate to only the minium needed network utilization between RRH and BBU 
can be greatly reduced compared from worst to best case.\\
There is only one implementation for software defined LoRa decoding that worked for us but it is fairly completed.
However the modulation of an uplink signal is still in the early stages. We extended the modulation by 
implementing the modulation of a downlink signal though but the correctness and sophistication is strictly tied to 
the uplink modulation implementation as the downlink signals is essentially the inverse of an uplink signal in terms of 
chirp direction.
\\
Having also modulation of signals defined in software, not only the decoding, would make prototyping and simulation
of network conditions much simpler. The difficulty lies in the fact that the LoRa PHY is proprietary which is why reverse 
engineering it is so hard. Having said that, as there already exists a functioning decoder, an encoder should not be that far 
away. We contribute some tools we developed during to this thesis such as the downlink modulator and chirp visualizer that could
possibly help for further development. 