\chapter*{Abstract}
\addcontentsline{toc}{chapter}{Abstract}

\selectlanguage{german}

Thema dieser Arbeit ist das Design und die Implemtierung einer Cloud Radio Access Networks (C-RAN)
Architektur für LoRa. LoRa ist eine weit verbreite Modulierungstechnik basierend auf dem Chirp Spread Spectrum (CSS)
Modulierungsverfahren. Es wird of benutzt in Low Power Wide Area Networks für Internet of Things Geräte. Herkömmlicherweise
empfangen Gateways in LoRa Wide Area Networks ein uplink Signal von Endgeräten und demodulieren und dekodieren das Signal
auf dem Gateway selbst. Für downlink Signal empfangen die Gateways eine digitale Nachricht von einem Netzwerk Server. Diese 
Nachricht wird dann von den Gateways moduliert und enkodiert als ein anologes Signal, dass dann über die Luft als Radiowellen
zu den Endgeräten gesendt wird. Die Gateways können jedoch vereinfacht werden indem das Demodulieren und Dekodieren, bzw. das Moduliern und Enkodieren
in Software implementiert wird welche auf Allzweck Hardware läuft. Dies ist die Baseband Unit (BBU). Das gateway fungiert dann 
lediglich noch als Remote Radio Head (RRH). Die BBU's können auf einem Server zentralisiert und virtualiesert werden.
Die C-RAN Architektur in dieser Arbeit ist mit einem Software Defined Radio (SDR) und mit Docker für die Virutalisierung der BBUs implementiert.
Desweiteren untersucht diese Arbeit Netzwerk Anforderungen und die Auswirkungen von Netzwerk -und Verarbeitungsverzögerung auf das C-RAN.
Ausserdem stellt diese Arbeit einen Softwaremodulator für downlink LoRa signal vor die von echter Hardware empfangen werden können.


\selectlanguage{english}
The topic of this thesis is the design and implementation of a Cloud Radio Access Network (C-RAN)
architecture for LoRa. LoRa is a popular modulation scheme based on the chirp spread spectrum modulation technique. It is 
used in Low Power Wide Area Networks for Internet of Things devices. Traditionally, in a LoRa Wide Area Network, gateways receive
signals on the uplink from end-devices and demodulated and decode the analog signal directly on the gateway itself.
On the downlink, gateways receive a digital message from the network server and modulated and encode this message 
as an analog signal and send it over the air as radio waves to the end-devices.
The gateways however can be simplified by moving the demodulation and decoding, resp. modulation and encoding,
from the gateway to general purpose hardware and implement it in software as a Baseband Unit (BBU). This leaves the gateway 
as a Remote Radio Head (RRH). The BBUs then can be centralized and virtualized on a server.
The C-RAN architecture in this thesis is implemented with a Software Defined Radio (SDR) and Docker for the virtualization of 
the BBU's. Next to designing and implementing the LoRa C-RAN, this thesis studies the network requirements and the effects 
of network and processing delay on the C-RAN. This thesis also introduces a software modulator and encoder for a downlink LoRa
signal that can be received by real hardware. 


