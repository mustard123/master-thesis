\chapter*{Abstract}
\addcontentsline{toc}{chapter}{Abstract}


\selectlanguage{english}
The topic of this thesis is the design and implementation of a Cloud Radio Access Network (C-RAN) architecture for Long Range (LoRa) networks. 
LoRa is a popular modulation scheme based on the chirp spread spectrum modulation technique. 
It is used in Low Power Wide Area Networks (LP-WANs) for Internet of Things (IoT). 
Traditionally, in a LoRa Wide Area Network (LoRaWAN), gateways receive signals on the uplink from end-devices to demodulate and decode the analog signal directly on the gateway.
On the downlink, gateways receive a digital message from the network server and modulate and encode this message as an analog signal to send it over the air as radio waves to end-devices.
The gateways, however, can be simplified by moving the demodulation and decoding in the receiver (resp. modulation and encoding in the sender) from the gateway to general purpose hardware running a software Base-Band Unit (BBU). 
This leaves the gateway as a Remote Radio Head (RRH) implementing only limited capabilities, while the BBUs can be centralized and virtualized on a remote server.
In this thesis, the C-RAN architecture is implemented with the help a Software Defined Radio (SDR) for handling radio signals as well as Docker for the virtualization of BBUs. 
To design and implement the LoRa C-RAN, this thesis studies the network requirements and the effects of network and processing delay on the C-RAN.
Finally, this work also also introduces a software modulator and encoder for a LoRa to emit signals that can be received by real hardware on the downlink. 

\newpage

\selectlanguage{german}

Thema dieser Arbeit ist das Design und die Implemtierung einer Cloud Radio Access Networks (C-RAN)
Architektur für LoRa. LoRa ist eine weit verbreite Modulierungstechnik basierend auf dem Chirp Spread Spectrum (CSS)
Modulierungsverfahren. Es wird of benutzt in Low Power Wide Area Networks für Internet of Things Geräte. Herkömmlicherweise
empfangen Gateways in LoRa Wide Area Networks ein uplink Signal von Endgeräten und demodulieren und dekodieren das Signal
auf dem Gateway selbst. Für downlink Signal empfangen die Gateways eine digitale Nachricht von einem Netzwerk Server. Diese 
Nachricht wird dann von den Gateways moduliert und enkodiert als ein anologes Signal, dass dann über die Luft als Radiowellen
zu den Endgeräten gesendt wird. Die Gateways können jedoch vereinfacht werden indem das Demodulieren und Dekodieren, bzw. das Moduliern und Enkodieren
in Software implementiert wird welche auf Allzweck Hardware läuft. Dies ist die Baseband Unit (BBU). Das gateway fungiert dann 
lediglich noch als Remote Radio Head (RRH). Die BBU's können auf einem Server zentralisiert und virtualiesert werden.
Die C-RAN Architektur in dieser Arbeit ist mit einem Software Defined Radio (SDR) und mit Docker für die Virutalisierung der BBUs implementiert.
Desweiteren untersucht diese Arbeit Netzwerk Anforderungen und die Auswirkungen von Netzwerk -und Verarbeitungsverzögerung auf das C-RAN.
Ausserdem stellt diese Arbeit einen Softwaremodulator für downlink LoRa signal vor die von echter Hardware empfangen werden können.

\selectlanguage{english}

